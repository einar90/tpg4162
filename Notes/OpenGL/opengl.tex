\documentclass[a4paper, english, 12pt]{article}

% General packages
\usepackage{parskip}      % Vertical space on new paragraph instead of indent
\usepackage{hyperref}   % For including links
\usepackage{url}          % Allows adding url's
\usepackage[top=4cm, bottom=4cm]{geometry}     % Easier management of page dimensions
\usepackage{chngcntr}     % Customizable counters
\usepackage{enumitem}     % Can use [noitemsep] on lists to reduce spacing
\usepackage{fancyhdr}
\pagestyle{fancy}
\usepackage{afterpage} 	% Various tweaks for page breaks and stuff

% BEGIN Floats
\usepackage[bf, hang, small]{caption}	% Nicer captions for floats.
\usepackage{float}        % Allows advanced placement options for floats
% \usepackage{subfig}		% Sub-figures. Crashes with captions package
% END Floats


% BEGIN Tables
\usepackage{booktabs}		% More professional looking tables
\usepackage{tabularx}		% Allows width adjustment etc. for tables
\usepackage{multirow}		% Allows table cells to span rows or columns




% Font specific packages
\usepackage{fontspec}
\usepackage{textcomp}
\usepackage{titlesec}
\usepackage{titling}
\usepackage{mathpazo}
\usepackage{inconsolata}

% Specifying fonts
\setsansfont{Open Sans}
\fontspec{Inconsolata}
\setmonofont[Scale=0.85]{Inconsolata}
\newfontfamily\headingfont[]{Open Sans}
\newfontfamily\headingfont[]{Open Sans}
\titleformat*{\section}{\Large\headingfont}
\titleformat*{\subsection}{\large\headingfont}
\titleformat*{\subsubsection}{\normalsize\headingfont}
\renewcommand{\maketitlehooka}{\headingfont}



\usepackage{moreverb}     % Allows use of tables in verbatim
\usepackage{fancyvrb}     % Fancy verbatim stuff
\usepackage{listings}     % Source code listings
\usepackage[usenames,dvipsnames]{color}		% Used for colors by various other packages

% Configuring listings
\definecolor{light-gray}{RGB}{250,250,250} 	% Background color for listings
\definecolor{gray}{RGB}{100,100,100} 	% Color for line-numbers
\lstset{
	language={C},			% the language of the code
	aboveskip=1em,
	belowskip=-3em,
	basicstyle=\ttfamily\footnotesize,
	backgroundcolor=\color{light-gray},
	frame=single,				% adds a frame around the code
	rulecolor=\color{black},	% frame color
	captionpos=b,				% sets the caption-position to bottom
	breaklines=true,			% sets automatic line breaking
	breakatwhitespace=false,	% automatic breaks only at whitespace?
	keepspaces=true,			% keeps spaces in text, useful for keeping indentation of code (possibly needs columns=flexible)
	tabsize=4,					% sets default tabsize to 2 spaces
	commentstyle=\color{green},	% comment style
	keywordstyle=\color{blue},	% keyword style
	stringstyle=\color{red},	% string literal style
	numbers=left,				% line-number position: none, left or right
	numbersep=5pt,				% how far the line-numbers are from the code
	numberstyle=\tiny\color{gray}, % the style that used for line-numbers
	stepnumber=2,				% step between two line-numbers.
	showspaces=false,			% show spaces everywhere with underscores
	showtabs=false,				% show tabs within strings with underscores
	showstringspaces=false,		% underline spaces within strings only
	escapeinside={\%*}{*)},		% if you want to add LaTeX within your code
	title=\lstname				% show the filename of files included with \lstinputlisting; also try caption instead of title
}





\title{OpenGL in C\\ 
\vspace{6pt}\large A short summary of OpenGL programming in the C language on Linux \\ 
 }
\author{Einar Baumann}   
\date{\today}




\begin{document}

\maketitle
\thispagestyle{empty}
\pagebreak

\section{General program structure}
\label{sec:general_structure}

\subsection{The old way}
\label{sub:old_way}
The old structure for drawing with OpenGL is something like the following:

\begin{lstlisting}
glBegin(GL_QUADS);	
	glColor3f(1.0f, 1.0f, 1.0f);	// set red, green, blue
	glVertex3f(-0.8f, -0.8f, 0.0f);
	glVertex3f(0.8f, -0.8f, 0.0f);
	glVertex3f(0.8f, 0.8f, 0.0f);
	glVertex3f(-0.8f, 0.8f, 0.0f);
glEnd();
\end{lstlisting}

% subsection old_way (end)

\subsection{The new way}
\label{sub:new_way}

\begin{lstlisting}
glClear(GL_COLOR_BUFFER_BIT);
const float coords[] = {
	-0.8f, -0.8f, 0.5f,	  
	 0.8f, -0.8f, 0.5f,   
	 0.8f,  0.8f, 0.5f,    
	-0.8f,  0.8f, 0.5f,  
};
const unsigned char indeces[] = { 0, 1, 2, 0, 2, 3 };

glEnableClientState(GL_VERTEX_ARRAY);
glVertexPointer(3, GL_FLOAT, 0, coords);
glColor3f(1.0f, 1.0f, 1.0f);	// set red, green, blue
glDrawElements(GL_TRIANGLES, 6, GL_UNSIGNED_BYTE, indeces);
\end{lstlisting}

% subsection new_way (end)

% section general_structure (end)


\section{Libraries to include in Linux}
\label{sec:linux_libs}

\begin{lstlisting}
#include<stdio.h>
#include<stdlib.h>
#include<X11/X.h>		// X window system
#include<X11/Xlib.h>	// Libraries for X window system
#include<GL/gl.h>		// Header file for OpenGL32 Library
#include<GL/glx.h>		// OpenGL Extension for X window system
#include<GL/glu.h>		// Header File For The GLu32 Library
\end{lstlisting}

% section linux_libs (end)


\section{OpenGL command formats}
\label{sec:command_formats}

Lets look at an example function: \texttt{glVertex3fv( v )}. There are a number of interesting points here:

\begin{itemize}
	\item \texttt{gl} is a prefix for all OpenGL functions.
	\item \texttt{Vertex} means that we're drawing a vertex.
	\item \texttt{3} means that the vertex has three components.
	\item \texttt{f} indicates that the components will have the datatype \texttt{float}.
	\item \texttt{v} indicates that the components will be given as a vector argument. The \texttt{v} may be omitted to use the scalar for instead, i.e. \texttt{glVertex3f( x, y, z )}. 
\end{itemize}

The possible values for ''number of components`` and ''datatype`` are given in, respectively, tables \ref{tab:number_of_components} and \ref{tab:data_types}.

\begin{table}
\label{tab:data_types}
\caption{Possible datatypes for OpenGL commands.}
\centering
\begin{tabular}{ll}
\toprule
	b	& Byte \\
	ub 	& Unsigned byte \\
	s 	& Short \\
	us 	& Unsigned short \\
	i 	& Integer\\
	ui 	& Unsigned integer\\
	f 	& Float\\
	d 	& Double \\
\bottomrule
\end{tabular}
\end{table}

\begin{table}
\label{tab:number_of_components}
\caption{Possible values for number of components in OpenGL commands}
\centering
\begin{tabular}{ll}
\toprule
	2	& (x, y) \\
	2 	& (x, y, z) \\
	4 	& (x, y, z, w) \\
\bottomrule
\end{tabular}
\end{table}

% section command_formats (end)


\section{OpenGL functions}
\label{sec:functions}

\begin{description}
	\item[\texttt{glVertexPointer(size, type, stride, pointer)}] Allows OpenGL to extract positional data from varous array and memory constructs. Has to be initialized using \texttt{glEnableClientState(GL\_VERTEX\_POINTER)}. The four parameters are:
	 \begin{itemize}[noitemsep]
		\item \texttt{size}: 2, 3 or 4; pads to homogenous coordinates.
		\item \texttt{type}: Usually \texttt{GL\_FLOAT}.
		\item \texttt{stride}: Byte offset between coordinate info in the array; or 0 for tightly packed array.
		\item \texttt{pointer}: A pointer to the data(array).
	\end{itemize}
	
	\item[\texttt{glDrawArrays(mode, first, count}] Draws geometric entities based on given array info. Positional data is retrieved from the \texttt{glVertexPointer} call; any other array info is retrieved as well. The function may inflict repetition of vertex and other information. The three parameters are:
	\begin{itemize}[noitemsep]
		\item \texttt{mode}: What to draw, e.g. \texttt{GL\_QUADS}, \texttt{GL\_TRIANGLES}.
		\item \texttt{first}: The first index.
		\item \texttt{count}: Number of indeces.
	\end{itemize}
	
	\item [\texttt{glDrawElements(mode, count, type, indeces}] Same as draw-arrays, but more explicit by allowing an array of indeces. Example: \texttt{glDrawElements(GL\_TRIANGLES, 6, GL\_UNSIGNED\_BYTE, indeces);}
	
	\item \texttt{glColorPointer(size, type, stride, pointer} Very similar to \texttt{glVertexPointer}. Must be enabled using \texttt{glEnableClientState(GL\_COLOR\_POINTER)}.
\end{description}

% section functions (end)

\end{document}













































